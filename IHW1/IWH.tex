\documentclass[amstex,11pt,a4paper]{article}
\usepackage[utf8]{inputenc}
\usepackage[T2A]{fontenc}

\usepackage[english,russian]{babel}
\usepackage{amsfonts,amssymb}
\usepackage{graphicx}
\usepackage{amsmath}
\usepackage{amsthm}
\usepackage{relsize}

\usepackage{systeme}

\usepackage{indentfirst} % Красная строка
\usepackage{fancyhdr}
\usepackage{wrapfig}
\usepackage{textcomp}

\usepackage{xcolor}% http://ctan.org/pkg/xcolor
%\usepackage{colortbl}% http://ctan.org/pkg/colortbl
\usepackage{multirow}% http://ctan.org/pkg/multirow
\usepackage{graphicx}% http://ctan.org/pkg/graphicx

\usepackage[unicode]{hyperref}
\usepackage{xcolor}

\definecolor{linkcolor}{HTML}{0000E6}
\definecolor{urlcolor}{HTML}{0000E6}
\definecolor{citecolor}{HTML}{0000E6}


\hypersetup{pdfpagemode=None,linktoc=page,citecolor=citecolor,linkcolor=linkcolor,urlcolor=urlcolor,colorlinks=true}

\addtolength{\textwidth}{130pt}
\addtolength{\hoffset}{-2cm}
\addtolength{\voffset}{-2cm}
\addtolength{\textheight}{90pt}

\tolerance=3000
\def\baselinestretch{1.1}
\flushbottom

\parindent=1cm

\theoremstyle{definition}
\newtheorem{problem}{Задача}
\newtheorem{ex}[problem]{Задача}
\newtheorem{disc}{Условие}
\newtheorem{resh}{Решение:}
\newtheorem{pl}{Первый Способ:}
\newtheorem{pk}{Второй Способ:}
\newtheorem{answer}{Ответ:}


%в новом разделе нумеруем заново
\makeatletter
\@addtoreset{problem}{section}
\makeatother

%Параметр от 0 до 9, иначе работать не будет.
\newcommand{\atten}[1]{\textcolor{red}{+#1 внимательность}}
\newcommand{\know}[1]{\textcolor{teal}{+#1 знания}}
\newcommand{\brave}[1]{\textcolor{violet}{+#1 храбрость}}
\newcommand{\wait}[1]{\textcolor{blue}{+#1 терпение}}
\newcommand{\bruteforce}[1]{\textcolor{brown}{+#1 грубая сила}}
\newcommand{\beauty}[1]{\textcolor{violet}{+#1 чувство прекрасного}}

\usepackage{tikz}

\begin{document}
%\vspace{-15pt}

\centerline{\textbf{ИДЗ 1}}
\centerline{\textbf{Вариант 4}}
\begin{flushright}
\emph{Бакиров Даниал Жандарбекович БПИ 226}
\end{flushright}

\thispagestyle{empty}

\section{Задачи}


\begin{ex}
\begin{disc}
    Из колоды карт (52 карты) наудачу извлекаются 3 карты. Определить
вероятность того, что это будут тройка, семерка, туз.
\\
\\
\begin{resh}
    \begin{pl}
        Можно рассмотреть решение с вереоятностью, где мы забираем карты из общего числа. То есть, все имеются 4 карты на которых есть троейка, 4 карты на которых есть семерка и 4 карты на которых есть туз. Всего карт у нас 52, а нужных нам карт 12 => $\frac{12}{52} $ - это вероятность выбора тройки. $\frac{12}{52} $ - это вероятность выбора семерки. $\frac{4}{50} $ - это вероятность выбора туза. 
        Тогда общая вероятность: 
        \begin{center}
        $\frac{12}{52} $ $\cdot$ $\frac{12}{52}$ $\cdot$ $\frac{4}{50} $ $\simeq$ $0,0028$
        \end{center}
    \end{pl}
    \begin{pk}
        Всего элементов будет $C_{52}^3$. Это так т.к. нам нужно выбрать 3 карты из 52 и мы считаем количество способов. $C_{52}^3 = 22100$. В числителе стоит выражение $4 * 4 * 4 = 64$ перемножаем т.к. вероятности не зависимы
        Тогда общая вероятность: 
        \begin{center}
        $\cdot$ $\frac{4 * 4 * 4}{C_{52}^3} = \frac{64}{22100}$ $\simeq$ $0,0028$
        \end{center}
    \end{pk}
    \begin{answer}
        Вероятность что извлекут тройку, семерку и туз равна: 0.0028.
    \end{answer}
\end{resh}
\end{disc}

\begin{disc}
    Счетчик регистрирует частицы трех типов: А, В и С. Вероятность
появления этих частиц такова: Р(А)=0.2, Р(В)=0.5, Р(С)=0.3. Частицы каждого из этих
типов счетчик улавливает с вероятностью: Р1 =0.8, Р 2 =0.2, Р 3 =0.4. Счетчик отметил
частицу. По критерию наибольшей вероятности определить, какая это была частица.
\\
\\
\begin{resh}
    В этой Задаче воспользуемся формулами полной вероятность и формула Байеса. Почему я это использую? Потому что у на должно произойти событие при условии, что произошло другое. То есть в нашем случае, у нас должна быть уловленна некая частица. То есть, мы уловили частицу при условии что она была первая, вторая, третья и т.д. Одним словом, как поступление новых данных о событии влияет на вероятность исхода этого события.
    Сначала подсчитаем полную вероятность. Обозначим это буквой D.
    \begin{center}
    P(D) = $P(A) * P_1 + P(B) * P_2 + P(C) * P_3 = 0.2 * 0.8 + 0.5 * 0.2 + 0.3 * 0.4 = 0.38$
    \end{center}
    Далее, по формуле Байеса подсчитаем вероятности событий и сравним их.
    Формула Байеса.\\ 
    Вероятость, что мы уловили первую частицу: 
    \begin{center}
    $P(P_1|A) = \frac{P(A|P_1) * P(P_1)}{P(D)} \Rightarrow P(P_1|A) = \frac{0.2 * 0.8}{0.38} = \frac{8}{19}$ 
    \end{center}
    Вероятость, что мы уловили вторую частицу: 
    \begin{center}
    $P(P_2|B) = \frac{P(B|P_2) * P(P_2)}{P(D)} \Rightarrow P(P_2|B) = \frac{0.5 * 0.2}{0.38} = \frac{5}{19}$ 
    \end{center}
    Вероятость, что мы уловили третью частицу: 
    \begin{center}
    $P(P_3|C) = \frac{P(C|P_3) * P(P_3)}{P(D)} \Rightarrow P(P_3|C) = \frac{0.3 * 0.4}{0.38} = \frac{6}{19}$ 
    \end{center}
    Далее мы видим, что вероятность улавливания первой частицы больше, значит, ее уловили первой.
    \begin{answer}
        Наибольшая вероятность у первой частицы.
    \end{answer}
\end{resh}
\end{disc}
\end{ex}


\end{document} 